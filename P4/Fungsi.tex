\chapter{Fungsi (Subprogram)}
\section{Tujuan}
\begin{itemize}
    \item Students understand how to create and call functions in C %Mahasiswa mengerti cara membuat dan memanggil fungsi pada bahasa pemrograman C
    \item Students able to pass parameter by value and by reference in C.%Mahasiswa mampu menggunakan passing parameter by value dan by reference pada bahasa pemrograman C.
    \item Students understand and able to apply recursion in C. %Mahasiswa mampu mengerti dan mengaplikasikan konsep rekursi pada bahasa pemrograman C.
    
\end{itemize}   

The advantages of using functions in C programming language are:
\begin{itemize}
	\item Some code snippets can be reusable when using functions. 
\item We can call C functions any number of times in a program and from any place in a program.
\item Large C codes can be splitted to several function, thus easier to track.%Program c yang besar dapat dibagi ke dalam beberapa fungsi sehingga dapat dengan mudah untuk dilacak.
\end{itemize}
\section{Function Declaration}
 Every C program has atleast one function, that is the main() function. You can also define functions other than main().
 
Syntax :
\begin{verbatim}
    return_type function_name( parameters list){
        // function body
    	return something;
    }
\end{verbatim}
\begin{itemize}
	\item Return Type.\\ The data type a function has to return.
	\item \verb*|function_name|.\\ The name of the function
	\item parameters list.\\ %Nilai atau argumen yang menjadi input parameter. Urutan nilai yang dimaskuak ke fungsi berurutan sesuai dengan parameter yang dimasukan ke fungsi.
    The parameters of the function. 
	\item Badan fungsi.\\ The block of code that will be executed when the function is called.%Kumpulan statemen yang mendefinisikan apa yang dilakukan oleh fungsi.
	\item \verb|return something;|\\ A statement to return a value (\verb|something|). Returning the function causes the function to end.%merupakan statement untuk mengembalikan nilai dari fungsi. Untuk fungsi yang tidak mengembalikan nilai, dapat digunakan \verb|return_type| \verb|void|. Untuk keluar dari fungsi itu hanya perlu menggunakan statement \verb*|return|
    For functions that doesn't return a value (\verb|void| type function), ending the function can be done by using \verb|return;|
\end{itemize}
Example

\begin{lstlisting}[language=c]
float TriangleArea(float Base, float Height)
{
	float Area;
	Area = 0.5*Base*Height;
	return Area;
}
\end{lstlisting}


\section{Calling a Function} 

\begin{figure}[H]
	\centering
	\includegraphics[width=0.45\linewidth]{Fungsi/screenshot005}
	\caption{}
	\label{fig:memanggilfungsi}
\end{figure}

\begin{lstlisting}[language=c]
	#include <stdio.h>
// Mendeklarasikan fungsi luasSegitiga
// Parameter input ALas , dan Tintgi
// Output float
float TriangleArea(float Base, float Height)
{
	float Area;
	Area = 0.5*Base*Height;
	return Area;
}
int main()
{
	float Bs = 4,Hg=10,L;
    // Calling the TriangleArea function
	L=TriangleArea(Bs,Hg);
	printf("Area = %f",L);
	return 0;
}
\end{lstlisting}
\begin{enumerate}
	\item Line 5-10: Defining the function \verb|Triangle Area| with %Mendefunisikan fungsi \verb*|TriangleArea| dengan 
	\begin{itemize}
		\item Two input parameter :\\
	input \verb*|Base| and \verb*|Height|  with \verb*|float| data type.
	    \item Single valued output with \verb*|float| data type.
    \end{itemize}
\end{enumerate}
\section{Function with Arguments}
\subsection{Arguments}


%Jika suatu fungsi diharapkan untuk menggunakan argumen, maka variabel sebagai parameter yang menerima nilai dari argumen tersebut harus di dedeklarasikan terlebih dahulu. \\
\begin{enumerate}
	\item  \textbf{Parameter :}
\begin{enumerate}
	%\item Parameter adalah variabel dalam fungsi untuk merujuk ke salah satu bagian dari
	%data yang diberikan sebagai input ke fungsi.
    \item Parameters are the variable in the function that points to the part of the data that is inputted to the function.
	%\item Data ini disebut argumen.
    \item These data is called arguments.
	
\end{enumerate}

\item \textbf{Formal Parameter:}
\begin{enumerate}
	%\item Parameter yang Ditulis dalam Definisi Fungsi Disebut “Parameter Formal
    \item Parameter that is written within the function definition is called formal parameter.
    \item Formal Parameter is always a variable, Actual Parameter however doesn't necessarily has to be a variable.
	%\item Parameter formal selalu variabel, sedangkan parameter aktual tidak harus variabel.
	
\end{enumerate}


\item \textbf{Actual Parameter:}
\begin{enumerate}
	%\item Parameter yang Ditulis ketika memanggil fungsi
    \item Parameter that is used when calling the function
    \item Actual Parameter could take the form of number, expression, or another function call.
	%\item Dapat berupa angka, ekspresi, atau bahkan panggilan fungsi.
\end{enumerate}
\begin{figure}[H]
	\centering
	\includegraphics[width=0.7\linewidth]{Fungsi/screenshot006}
	\caption{}
	\label{fig:parameterformalaktual}
\end{figure}
\end{enumerate}
\subsection{Parameter Passing} 
%Passing parameter merupakan aktivitas menyalurkan nilai pada parameter saat memanggil fungsi. Pada umumnya, dikenal dua macam passing parameter yaitu:
To use a function with parameter, the parameters must be passed to the function first.
In general, there are two ways to pass paramater to a function
\begin{itemize}
	\item Pass parameter by value, pass the value of the variable to the function.%yaitu menyalurkan \textbf{nilai} dari tiap parameter yang diberikan.
	\item Pass parameter by reference, pass the reference of a variable (its memory address) to the function. %yaitu menyalurkan \textbf{alamat} dari tiap parameter yang diberikan.
\end{itemize}

\subsubsection{Passing Parameter by Value}

\begin{lstlisting}[language=c,caption = Passing by Value,label=lst:passbyvalue01]
#include<cstdio>
int swapAndReturnTheSum(int x, int y) {
    int z;
    z = x;
    x = y;
    y = z;
    return x+y;
}
int main()
{
    int a = 1;
    int b = 2;
    int sum = swapAndReturnTheSum(a,b);
    printf("sum: %d\n",sum);
    printf("values of a dan b now:\n");
    printf("a: %d\n",a);
    printf("b: %d\n",b);
}
\end{lstlisting}

%Perhatikan potongan kode pada Listing \ref{lst:passbyvalue01}. Baris 3-6 dari kode tersebut adalah operasi untuk menukar nilai dari 2 variabel. Namun, apabila program tersebut dijalankan, maka akan muncul output
line 3-6 of the code in Listing \ref{lst:passbyvalue01} is a set of assignments to swap the values of 2 variable. However, when the program is executed, the output would be the following.
\begin{verbatim}
    sum: 3
    values a and b now:
    a: 1
    b: 2
\end{verbatim}
The values of a and b did not swap. When passing parameter by value, anything that is done within the function body will have no effect on the parameter that is "passed on" the function. The value of the actual parameter will be assigned to the formal parameter, so we are not doing operation directly on the actual parameter.
%Nilai dari a dan b tidak bertukar. Untuk passing parameter by value, apapun yang dilakukan pada function body tidak akan berpengaruh pada parameter yang "dipassingkan". Nilai dari parameter aktual akan diassign pada parameter formal.

\subsubsection{Passing Parameter by Reference}
%Perhatikan baris 2 pada potongan kode berikut:
Look at line 2 of the following code.
\begin{lstlisting}[language=c,caption = Passing by Reference,label=lst:passbyreference01]
#include<cstdio>
int swapAndReturnTheSum(int &x, int &y) {
    int z;
    z = x;
    x = y;
    y = z;
    return x+y;
}
int main()
{
    int a = 1;
    int b = 2;
    int sum = swapAndReturnTheSum(a,b);
    printf("sum: %d\n",sum);
    printf("values of a dan b now:\n");
    printf("a: %d\n",a);
    printf("b: %d\n",b);
}
\end{lstlisting}
%apabila program tersebut dijalankan, maka akan muncul output
When this program is executed, it will output the following.
\begin{verbatim}
    sum: 3
    values of a and b now
    a: 2
    b: 1
\end{verbatim}

%Ketika fungsi \verb|swapAndReturnTheSum(a,b)| dipanggil, alamat memori variabel a dan b "dipassingkan" pada fungsinya. Sehingga pada pada potongan kode di baris 4-6, x dan y akan mengacu pada memori parameter aktual yang dimasukkan di baris ke 13. Ketika melakukan passing by reference, kita tidak bisa memanggil fungsi dengan parameter yang tidak memiliki alamat memori. Sebagai contoh \verb|tukarDanKembalikansumnya(1,2)| tidak bisa dilakukan karena angka 1 dan 2 bukan variabel dan tidak memiliki alamat memori.
When the function \verb|swapAndReturnTheSum(a,b)| is called, the memory address of \verb|a| and \verb|b| is passed on to the function. Therefore, in line 4-6, the \verb|x| and \verb|y| will point to the memory of the actual parameter that is inputted in line 13, so we are doing assignments directly to the actual parameter. When passing by reference, we can't call the function with parameter that has no memory address. As example \verb|swapAndReturnTheSum(1,2)| cannot be done as the number 1 and 2 doesn't have memory address.

\section{Recursion}
%Rekursi adalah ketika suatu fungsi dalam function bodynya memanggil fungsi itu sendiri.
%Sebagai contoh, perhatikan potongan kode berikut:
Recursion is when a function calls itself within its function body.
As an example, look at the code below.
\begin{lstlisting}[language=c,caption = Factorial with recursion,label=lst:recursionexample01]
int factorial(int n) {
    if (n==1)
        return 1;
    return n*factorial(n-1);
}
\end{lstlisting}
The factorial function calls another factorial function in line 4.
Initialy, the function $factorial(n)$ is called. This function however will return 
$n\times factorial(n-1)$, then $factorial(n-1)$ akan mengembalikan $(n-1)\times factorial(n-1-1)$.
Eventually it became like this:
%Dapat dilihat bahwa fungsi factorial pada function bodynya memanggil factorial pada baris 4.
%Pada awalnya jika fungsi $factorial(n)$ dipanggil maka dia akan mencoba untuk mengembalikan
\begin{equation*}
    \begin{split}
        factorial(n)& = n \times factorial(n-1)\\
        & = n \times (n-1) \times factorial(n-2)\\
        & = n \times (n-1) \times (n-2) \times \cdots \times 2 \times factorial(1)\\
        & = n \times (n-1) \times (n-2) \times \cdots \times 2 \times 1\\
    \end{split}
\end{equation*}

\section{Exercise}
\begin{itemize}
%    \item Buatlah fungsi yang dapat menerima 2 buah bilangan bulat a dan b kemudian mengembalikan nilai dari $a^b$
    \item Create a function that can take 2 integer a and b then returns $a^b$
%    \item Buatlah program dengan algoritma bubble sort tetapi proses penukaran 2 elemen pada array dilakukan dengan fungsi.
    \item Create program that do the bubble sort algorithm but the process of swapping 2 elements in the array is done with a function.
%    \item Masalah-masalah apa yang akan lebih mudah diselesaikan dengan menggunakan fungsi?
    \item What problems that can be solved easier with functions?
\end{itemize}
