\section*{Tujuan}
\begin{itemize}[label=$\bullet$, itemsep=-1pt, leftmargin=*]
    \item Mahasiswa mengerti tentang konsep pointer pada bahasa pemrograman C.
    \item Mahasiswa mengerti cara membuat dan memanggil struct pada bahasa pemrograman C.
    \item Mahasiswa mengerti tentang algoritma sorting pada bahasa pemrograman C.
    \item Mahasiswa mengerti tentang algoritma searching pada bahasa pemrograman C.
    \item Mahasiswa mampu mengaplikasikan konsep algoritma searching dan sorting pada bahasa pemrograman C.
\end{itemize}

\section{Pointer}

Pointer adalah sebuah variabel yang menyimpan alamat memori dari variabel lain.
Sebuah variabel yang dibuat di C menempati sebuah ruang di memori, dan ruang ini memiliki sebuah alamat biasa disebut sebagai alamat memori (memory address).
Untuk mendapat alamat memori kita bisa menggunakan operator \&. \\
Contoh:
\begin{lstlisting}[language=c]
	int a = 10;
	printf("%d\n", a); // output 10
	printf("%p\n", &a); // output alamat variabel a
\end{lstlisting}

\subsection{Deklarasi dan Inisialisasi}

Untuk mendeklarasikan pointer, sama seperti mendeklarasikan variabel dengan menambahkan * setelah tipe data
{
\captionsetup[lstlisting]{labelformat=empty, justification=raggedright, singlelinecheck=false} %agar caption tanpa label dan di kiri
\begin{lstlisting}[language=c, caption={syntax}]
	tipe_data *nama;
\end{lstlisting}
}
Untuk inisialisasi kita hanya perlu memasukkan nilai alamat dari variabel lain
contoh:
\begin{lstlisting}[language=c]
	int angka;
	int *ptr = &angka; // inisialisasi pointer ptr yang menunjuk ke alamat variabel angka

	printf("%p\n", &angka); // output alamat angka
	printf("%p\n", ptr);    // output alamat angka
\end{lstlisting}

\subsection{Dereference}

Dereference adalah istilah yang digunakan untuk mendapatkan nilai dari variabel yang ditunjuk oleh pointer menggunakan pointer.
Contoh:
\begin{lstlisting}[language=c]
	int angka = 10;
	int *ptr = &angka;

	printf("%d\n", angka); // output nilai angka
	printf("%d\n", *ptr);  // output nilai angka
\end{lstlisting}
\begin{verbatim}
	Output:
	10
	10
\end{verbatim}

\subsection{Pointer dengan variabel}

Dengan menggunakan pointer, kita bisa mengakses sebuah variabel tanpa memanggil variabel tersebut.
Kita hanya perlu memanggil pointernya.
Contoh:
\begin{lstlisting}[language=c]
	int angka = 5;
	int *ptr = &angka;

	printf("%d\n", angka);

	*ptr = 15; // akses variabel angka menggunakan pointer

	printf("%d\n", angka);  
\end{lstlisting}
\begin{verbatim}
    Output:
    5
    15
\end{verbatim}
Dari contoh diatas variabel angka diakses menggunakan pointer ptr sehingga nilai dari variabel angka berubah.

\subsection{Pointer dengan Array}

Pointer bisa digunakan juga untuk menyimpan alamat memori dari sebuah array.
Di C nama dari sebuah array menyimpan nilai dari alamat elemen pertamanya.
Contoh:
\begin{lstlisting}[language=c]
	int arr[5] = {10, 20, 30, 40, 50};
	printf("%p\n", arr);     // alamat elemen pertama (arr[0])
	printf("%p\n", &arr[0]); // alamat elemen pertama (arr[0])
\end{lstlisting}
Jadi bisa dibilang nama sebuah array merupakan pointer yang menunjuk ke elemen pertama array tersebut.
Contoh:
\begin{lstlisting}[language=c]
	int arr[5] = {10, 20, 30, 40, 50};
	printf("%d\n", arr[0]); // akses elemen pertama arr
	printf("%d\n", *arr);   // akses elemen pertama arr    
\end{lstlisting}
\begin{verbatim}
    Output:
    10
    10
\end{verbatim}
Untuk mengkases elemen selanjutnya, kita tinggal menambahkan pointer arr satu-persatu.
Contoh:
\begin{lstlisting}[language=c]
	int arr[5] = {10, 20, 30, 40, 50};
	printf("%d\n", *(arr+1)); // akses elemen kedua arr
	printf("%d\n", *(arr+2)); // akses elemen ketiga arr    
\end{lstlisting}
\begin{verbatim}
    Output:
    20
    30
\end{verbatim}
Kita juga bisa menggunakan loop:
\begin{lstlisting}[language=c]
	int arr[5] = {10, 20, 30, 40, 50};
	for(int i=0;i<5;i++){
		printf("%d\n", *(arr+i));
	}
\end{lstlisting}    
\begin{verbatim}
	Output:
	10
	20
	30
	40
	50
\end{verbatim}
Pointer juga bisa mengubah nilai dari array:
\begin{lstlisting}[language=c]
	int arr[3] = {1, 2, 3};
	int *ptr = arr; // pointer yang menunjuk ke arr

	*(ptr+1) = 4; // mengubah nilai arr index ke-1

	printf("%d\n", arr[1]);
\end{lstlisting}
\begin{verbatim}
	Output:
	4
\end{verbatim}

\subsection{Pointer dan fungsi}

Dengan menggunakan pointer kita bisa memasukkan variabel dari luar fungsi dan mengubah nilai dari variabel tersebut dari dalam fungsi.
Hal ini disebut sebagai passing parameter by reference, sedangkan jika memasukkan argumen seperti biasa disebut passing parameter by value.

\subsubsection{Passing by value}

Secara default, fungsi di C hanya mengirim nilai dari argumen ke parameter di dalam fungsi.
Jadi meskipun nilainya sama tapi variabelnya berbeda.
Contoh:

\begin{lstlisting}[language=c]
#include <stdio.h>

void ubah(int x) {
	printf("%d ", x);
	x = 20;   // mengubah variabel lokal x
	printf("%d\n", x);
}

int main() {
	int a = 10;
	ubah(a);
	printf("%d\n", a);
}
\end{lstlisting}
\begin{verbatim}
	Output:
	10 20
	10
\end{verbatim}
Dari contoh diatas, nilai dari variabel a tetaplah 10, meskipun variabel x berubah.
Hal ini karena hanya nilai 10 yang dikirim kedalam fungsi dan disalin ke variabel x, bukan variabel a nya sendiri.

\subsubsection{Passing by Reference}

Digunakan untuk mengubah nilai variabel di luar fungsi menggunakan pointer.
Contoh:
\begin{lstlisting}[language=c]
#include <stdio.h>

void ubah(int *x) { // parameter berupa pointer
	*x = 20;   // ubah nilai di alamat x
}

int main() {
	int a = 10;
	ubah(&a);   // kirim alamat a
	printf("%d\n", a);
}
\end{lstlisting}
\begin{verbatim}
	Output:
	20
\end{verbatim}
Dari contoh diatas, nilai a berubah meskipun perubahannya ada di dalam fungsi karena menggunakan passing by reference.
Hal ini bisa digunakan untuk membuat fungsi yang bisa menukar nilai dari dua variabel:
\begin{lstlisting}[language=c]
#include <stdio.h>

void swap(int *x, int *y) {
	int temp = *x;
	*x = *y;
	*y = temp;
}

int main() {
	int a = 5, b = 10;
	swap(&a, &b);
	printf("a = %d, b = %d\n", a, b);
}
\end{lstlisting}
\begin{verbatim}
	Output:
	a = 10, b = 5
\end{verbatim}
Passing by reference bisa digunakan juga untuk memasukkan sebuah array sebagai parameter fungsi.
Contoh:
\begin{lstlisting}[language=c]
void cetakArray(int *arr, int n) { // pointer sebagai parameter untuk array
	for (int i = 0; i < n; i++) {
	printf("%d ", *(arr + i));
	}
	printf("\n");
}

int main() {
	int data[5] = {1, 2, 3, 4, 5};
	cetakArray(data, 5); // memasukkan array ke fungsi
}
\end{lstlisting}
\begin{verbatim}
    Output:
    1 2 3 4 5     
\end{verbatim}

\subsection*{Tugas Pendahuluan 1}
\begin{enumerate}
    \item Apa saja kegunaan pointer? Jelaskan!
    \item Bagaimana cara mendeklarasikan pointer ke array multidimensi?
    \item Perhatikan kode berikut:
    \begin{lstlisting}[language=c]
#include <stdio.h>

int main() {
	int arr[4] = {25, 50, 75, 100};

	for (int i = 0; i < 4; i++) {
		printf("%d\n", *(arr++));
	}
}
\end{lstlisting}
    Kode diatas memiliki error, kenapa error itu terjadi? bagaimana solusinya?
    \item Jelaskan apa yang kamu ketahui tentang pointer ke pointer! Berikan contoh penggunaannya!
    \item Pelajarilah Algoritma sorting! Tuliskan apa saja yang kamu pelajari!

\end{enumerate}

\section{Struct}

Struct adalah sebuah cara untuk mengelompokkan beberapa variabel (bisa dari tipe yang berbeda) menjadi satu.
Jika array bisa menyimpan banyak data dengan tipe data yang sama, struct bisa menyimpan bermacam tipe data.
Contoh:
\begin{lstlisting}[language=c]
#include <stdio.h>

struct Mahasiswa {
	int nrp;
	char nama[50];
	float ipk;
};

int main() {
	struct Mahasiswa m1 = {5024991000, "B300", 3.99};

	printf("NRP  : %d\n", m1.nrp);
	printf("Nama : %s\n", m1.nama);
	printf("IPK  : %.2f\n", m1.ipk);

	return 0;
}
\end{lstlisting}

\subsection{Membuat struct}

Untuk membuat struct kita menggunakan keyword struct:
{
\captionsetup[lstlisting]{labelformat=empty, justification=raggedright, singlelinecheck=false} %agar caption tanpa label dan di kiri
\begin{lstlisting}[language=c, caption={syntax}]
	struct Coord {
		int x;
		int y;
	};
\end{lstlisting}
}
Jangan lupa untuk menambahkan titik-koma diakhir kurung kurawal.\\\\
Untuk mengakses struct, kita perlu mendeklarasi variabel untuk struct tersebut.
Dan untuk mengakses anggota dari sebuah struct kita panggil variabelnya dan tambahkan titik lalu nama anggotanya.
Contoh:
\begin{lstlisting}[language=c]
	struct Coord {
		int x;
		int y;
	};

	int main(){
		struct Coord p1 = {10, 20}; // membuat variabel struct

		printf("%d %d\n", p1.x, p1.y); // mengakses anggota p1

		p1.x = 30; // mengubah anggota p1
		printf("%d %d\n", p1.x, p1.y);
	}    
\end{lstlisting}
\begin{verbatim}
    Output:
    10 20
    30 20    
\end{verbatim}

\subsection{Struct dan pointer}

Pointer bisa digunakan untuk menunjuk ke struct, dan akses ke anggotanya menggunakan pointer dan oprator panah (->).
Contoh:
\begin{lstlisting}[language=c]
	struct Coord {
		int x;
		int y;
	};

	int main(){
		struct Coord p1 = {10, 20};
		struct Coord *ptr = &p1;
		
		printf("%d %d\n", ptr->x, ptr->y);
	}
\end{lstlisting}
\begin{verbatim}
    Output:
    10 20   
\end{verbatim}

\subsection{Array Struct}

Struct bisa dijadikan array.
Contoh:
\begin{lstlisting}[language=c]
	struct Mahasiswa {
		char nrp[20];
		char nama[50];
		float ipk;
	};
	
	int main() {
		struct Mahasiswa mhs[2] = {
			{"5024991001", "mahasiswa1", 3.1},
			{"5024991002", "mahasiswa2", 3.2}
		};
	
		for (int i = 0; i < 2; i++) {
			printf("%s %s %.2f\n", mhs[i].nrp, mhs[i].nama, mhs[i].ipk);
		}
	}
\end{lstlisting}
\begin{verbatim}
    Output:
    5024991001 mahasiswa1 3.10
    5024991002 mahasiswa2 3.20
\end{verbatim}

\subsection{Nested struct}

Nested struct adalah sebuah struct yang memiliki anggota berupa struct lain.
Hal ini akan berguna jika data yang disimpan memiliki hirarki.
Contoh:
\begin{lstlisting}[language=c]
#include <stdio.h>

// Struct alamat
struct Alamat {
	char jalan[50];
	char kota[30];
};

// Struct mahasiswa yang berisi struct alamat
struct Mahasiswa {
	char nrp[20];
	char nama[50];
	float ipk;
	struct Alamat alamat;  // nested struct
};

int main() {
	struct Mahasiswa m1 = {
		"5024991000", "B300", 3.8,
		{"Jalan Raya ITS", "Surabaya"}
	};

	printf("NRP    : %s\n", m1.nrp);
	printf("Nama   : %s\n", m1.nama);
	printf("IPK    : %.2f\n", m1.ipk);
	printf("Alamat : %s, %s\n",
		m1.alamat.jalan,
		m1.alamat.kota);

	return 0;
}
\end{lstlisting}
\begin{verbatim}
    Output:
    NRP    : 5024991000
    Nama   : B300
    IPK    : 3.80
    Alamat : Jalan Raya ITS, Surabaya
\end{verbatim}

\subsection{Typedef struct}

Dengan menggunakan typedef kita bisa membuat penulisan struct menjadi lebih singkat.
Contoh:
\begin{lstlisting}[language=c]
	typedef struct { // Membuat struct dengan typedef
		char nrp[20];
		char nama[50];
		float ipk;
	} Mahasiswa;

	int main(){
		Mahasiswa m1 = {"5024991000", "B300", 3.99}; // deklarasi variabel struct tanpa menulis struct Mahasiswa.

		printf("NRP  : %d\n", m1.nrp);
		printf("Nama : %s\n", m1.nama);
		printf("IPK  : %.2f\n", m1.ipk);
		
		return 0;
	}
\end{lstlisting}

\subsection*{Tugas Pendahuluan 2}
\begin{enumerate}
    \item Apakah struct bisa digunakan sebagai parameter fungsi? Bagaimana caranya?
    \item Apa itu union? apa perbedaannya dengan struct? Jelaskan!
    \item Jika terdapat tiga data nilai, yang berupa nilai tugas, nilai ETS, dan nilai EAS.
    Ketiga data nilai tersebut bisa disimpan sebagai integer.
    Menurutmu ketiga data tersebut lebih baik disimpan sebagai array atau struct? Jelaskan alasanmu!
\end{enumerate}