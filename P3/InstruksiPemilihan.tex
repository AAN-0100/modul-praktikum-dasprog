\chapter{Conditional Instruction}

\section{Goals}
\begin{itemize}
%    \setlength\itemsep{0.5em}
	%\item Mahasiswa mengenal dan mampu menggunakan ekspresi-ekspresi logika dan perbandingan pada bahasa pemrograman C
	\item Students know and able to utilize logical and comparison expression in C programming language.
	%\item Mahasiswa mengenal dan mampu menggunakan syntax-syntax percabangan pada bahasa pemrograman C
	\item Students know and able to utilize conditional instruction sytaxes in C programming language.
\end{itemize}


\section{Logical and Comparison Expression}

\subsection{Comparison Expression}
Berikut adalah operator-operator yang digunakan pada suatu ekspresi perbandingan
The following are the operators used in comparison expressions.
\begin{center}
    \captionof{table}{Comparison Operators \label{tab:operatorcomp}}
	\begin{tabular}{|c|l|c|}
		\hline
		Operator        & Name                                   & \multicolumn{1}{l|}{Example Expression} \\ \hline
		==              & Equals	                             & x == y                       \\ \hline
		!=              & Not Equals 	                         & x != y                       \\ \hline
		\textgreater{}  & Greater than                           & x \textgreater y             \\ \hline
		\textless{}     & Lesser than                            & x \textless y                \\ \hline
		\textgreater{}= & Greater or Equal than				 	 & x \textgreater{}= y          \\ \hline
		\textless{}=    & Lesser or Equal than           		 & x \textless{}= y             \\ \hline
	\end{tabular}
\end{center}

%Suatu ekspresi perbandingan akan mengembalikan nilai berupa \verb|true| atau \verb|false| yang ditandakan dengan nilai 0 atau 1.
A Comparison Expression will return boolean value \verb|true| or \verb|false| which is also represented with the value 1 or 0.
%Sebagai contoh:
As example:
\begin{verbatim}
    printf("%d",0>1); // will output 0 to the screen
    printf("%d",0<1); // will output 1 to the screen 
\end{verbatim}

\subsection{Logical Expression}
Berikut adalah operator-operator logika yang digunakan pada suatu ekspresi logika
The following are the logical operators used on a Logical Expression
\begin{center}
    \captionof{table}{Logical Expression \label{tab:operatorlogic}}
	\begin{tabular}{|c|l|c|}
		\hline
		Operator & \multicolumn{1}{c|}{Name} & Example Expression \\ \hline
		$\&\&$    &  AND                & $x<5\; \&\& \;x<10$\\ \hline
		$||$    &  OR                 & $x < 5\; ||\; x < 4  $     \\ \hline
		$!$        & NOT                & $!(x <5 \&\& x < 10) $\\ \hline
	\end{tabular}
\end{center}
%Sama seperti ekspresi perbandingan, ekspresi logika akan mengembalikan nilai berupa true atau false
Like comparison expression, logical expression will return boolean values.


\section{If statement}
\verb*|if| statement is used to decide which block of code to be executed if the condition is true.
%digunakan untuk menentukan blok kode C yang dijalankan apabila ekspresi kondisi bernilai benar (TRUE),
\begin{verbatim}
// Block of code before if
if (Condition) 
{
 // Block of code to be executed if the condition is true.
}
// block of code after if
\end{verbatim}
Sebagai contoh, perhatikan program berikut
As example, look at the following code 
\begin{lstlisting}[language=c,caption = If statement example,label=lst:ifexample01]
	include <stdio.h>
	
	int main()
	{
		//Deklarasi variabel 
		int uangSaya,hargaRoti;
		uangSaya = 5000;
		hargaRoti = 10000;
		
		if (uangSaya>=hargaRoti)
		{
		    printf("saya bisa beli roti\n");
		}
		printf("hehe");
		return 0;
	}
\end{lstlisting}                        
This program outputs
\begin{verbatim}
    hehe
\end{verbatim}
If line 7 changed to \verb|uangSaya=10000|, the outputs of the program would be
%Jika baris ke 7 diganti dengan \verb|uangSaya=10000| maka output dari program ini akan menjadi
\begin{verbatim}
    saya bisa beli roti
    hehe
\end{verbatim}

\section{If-else statement}
Statement else digunakan untuk menentukan blok kode yang di jalankan apabila kondisi salah. 
Else statement is used to decide the block of code to be executed if the condition is false.
\begin{verbatim}
// Block of code before if
if (Condition) 
{
	// Block of code to be executed if the condition is true
} else
{
	// Block of code to be executed if the condition is false
}
// Block of code after if-else statement
\end{verbatim}
%Berikut contoh penggunaan if-else
The following is an example of using if-else statement:
\begin{lstlisting}[language=c,caption = if-else example,label=lst:ifelseexample01]
	include <stdio.h>
	
	int main()
	{
		//Deklarasi variabel 
		int uangSaya,hargaRoti;
		uangSaya = 5000;
		hargaRoti = 10000;
		
		if (uangSaya>=hargaRoti)
		{
		    printf("saya bisa beli roti\n");
		}
		else
		{
	        printf("saya tidak bisa beli roti\n");	
		}
		printf("hehe");
		return 0;
	}
\end{lstlisting}                        
The output of the program is
\begin{verbatim}
    saya tidak bisa beli roti
    hehe
\end{verbatim}
%Jika baris ke 7 diganti dengan \verb|uangSaya=10000| maka output dari program ini akan menjadi
If line 7 changed to \verb|uangSaya=10000|, the outputs of the program would be
\begin{verbatim}
    saya bisa beli roti
    hehe
\end{verbatim}
\section{Statement if-else if}
Statement \verb|else if| digunakan untuk menjalankan blok kode apabila kondisi statement \verb|if| atau \verb|else if| sebelumnya bernilai salah.
The \verb|else if| statement is used to run a block of code when the condition in \verb|if| or the previous \verb|else if| is false.
\begin{verbatim}
	// block of code before if
    if (Condition1)
    {
	  /* block of code to be executed if Condition1
	  is true*/
    }
    else if (Condition2)
    {
	  /* block of code to be executed if Condition1 is false
	  and Condition2 is true */
    }
    else if (Condition3)
    {
	  /* Block of code to be executed when
	  Condition1 and Condition2 is false and
	  Condition3 is true*/
    }
    ...
    else if (ConditionN)
    {
	  /* Block of code to be executed when
	  Condition1 to ConditionN-1 is false and
	  ConditionN is true*/
    }
    else
    {
	  /* Block of code to be executed when
	  Condition1 to ConditionN is false*/
    }
	// Block of code after if
\end{verbatim}
Berikut contoh penggunaan if-else if
\begin{lstlisting}[language=c,caption = if-else if example,label=lst:ifelseifexample01]
	include <stdio.h>
	
	int main()
	{
		//Deklarasi variabel 
		int uangSaya,hargaRoti;
		uangSaya = 5000;
		hargaRoti = 10000;
		
		if (uangSaya>hargaRoti)
		{
		    printf("saya bisa beli roti\n");
		}
		else if(uangSaya==hargaRoti)
		{
		    printf("saya bisa beli roti tapi uang saya akan langsung habis\n");
		}
		else
		{
	        printf("saya tidak bisa beli roti\n");	
		}
		printf("hehe");
		return 0;
	}
\end{lstlisting}                        
The output of this program is
\begin{verbatim}
    saya tidak bisa beli roti
    hehe
\end{verbatim}
%Jika baris ke 7 diganti dengan \verb|uangSaya=10000| maka output dari program ini akan menjadi
If line 7 changed to \verb|uangSaya=10000|, the output of the program would be
\begin{verbatim}
    saya bisa beli roti tapi uang saya akan langsung habis
    hehe
\end{verbatim}
%Jika baris ke 7 diganti dengan \verb|uangSaya=12000| maka output dari program ini akan menjadi
If line 7 changed to \verb|uangSaya=12000|, the output of the program would be
\begin{verbatim}
    saya bisa beli roti
    hehe
\end{verbatim}

\section{Nested if}
%nested if merupakan konsep di mana di dalam suatu blok if terdapat statement if.
Nested if is when there is a conditional statements within a block of code inside the conditional statement
\begin{verbatim}
// Block of code before if
if (Condition1) 
{
    if (Condition2)
    {
        // do something
    }
    else
    {
        // do another thing
    }
} 
else
{
    // do something else
}
\end{verbatim}

%Berikut contoh penggunaan nested if
Below is an example of using nested if

\begin{lstlisting}[language=c,caption = nested if example,label=lst:nestedifexample01]
	include <stdio.h>
	
	int main()
	{
		// Declare the variables
		int myMoney,breadPrice,friendsMoney;
		myMoney = 5000;
		breadPrice = 10000;
		friendsMoney = 42069;
		
		
		if (myMoney>breadPrice)
		{
		    printf("I can buy bread\n");
		}
		else if(myMoney==breadPrice)
		{
		    printf("I can buy bread but I will ran out of money\n");
		}
		else
		{
		    if(friendsMoney+myMoney >= breadPrice)
		    {
		        printf("I can buy bread if I borrow my friend money\n"); 
		    }
		    else
		    {
	            printf("I can't buy bread\n");	
		    }
		}
		printf("hehe");
		return 0;
	}
\end{lstlisting}


\section{Exercise}
%Coba buat program yang menerima input 3 buah bilangan bulat A, B, dan C. Outputkanlah 3 bilangan bulat itu ke layar dengan urutan paling kecil ke paling besar. Lakukanlah ini dengan menggunakan statement if, if else, if else if, atau nested if.
Try to make a program that receives 3 integer input A, B, and C. Then outputs those 3 integers to the screen sorted from smallest to largest. Do this only using conditional statements.

