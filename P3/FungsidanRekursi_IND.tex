% \chapter{Fungsi (Subprogram)}
\section*{Tujuan}
\begin{itemize}[label=$\bullet$, itemsep=-1pt, leftmargin=*]
    % \item Students understand how to create and call functions in C .
    \item Mahasiswa mengerti cara membuat dan memanggil fungsi pada bahasa pemrograman C.
          % \item Students able to pass parameter by value and by reference in C.
    \item Mahasiswa mampu menggunakan passing parameter by value dan by reference pada bahasa pemrograman C.
          % \item Students understand and able to apply recursion in C. 
    \item Mahasiswa mampu mengerti dan mengaplikasikan konsep rekursi pada bahasa pemrograman C.
\end{itemize}

\section{Fungsi}

Fungsi adalah blok kode bernama yang dibuat untuk melakukan tugas tertentu.
Fungsi membantu membuat program lebih modular, rapi, dan mudah dipelihara.
Setiap fungsi dapat dipanggil berulang kali dari bagian program manapun tanpa menulis ulang kode.
\\\\Manfaat fungsi
\begin{enumerate}
    \item \textbf{Mengurangi Duplikasi Kode} \\
    Jika ada perintah yang sama dipakai berulang, cukup ditulis satu kali di dalam fungsi.
    Tinggal dipanggil kapanpun dibutuhkan.
    \item \textbf{Meningkatkan Keterbacaan dan Struktur Program} \\
    Program dengan fungsi lebih mudah dibaca dan dipahami karena tersusun rapi.
    Nama fungsi juga bisa mendeskripsikan tugasnya.
    \item \textbf{Memudahkan Pemeliharaan (Maintenance)} \\
    Jika ada perubahan, cukup memperbarui fungsi terkait tanpa harus mengubah seluruh kode program.
    \item \textbf{Dapat Digunakan Kembali (Reusability)} \\
    Fungsi bisa dipakai kembali di program lain dengan sedikit atau tanpa modifikasi.
    \item \textbf{Memudahkan Debugging dan Testing} \\
    Kesalahan lebih mudah dilacak karena fungsi bisa diuji satu per satu.
\end{enumerate}
Berdasarkan sumbernya, fungsi dibedakan menjadi dua, yaitu:
\begin{enumerate}
    \item \textbf{Fungsi library (Predefined function)} \\
    Fungsi yang sudah tersedia dalam bahasa C, seperti main(), printf() dan scanf().
    Fungsi ini berasal dari sebuah library yang dimasukkan dengan header file seperti \verb|#include <stdio.h>|.
    \item \textbf{Fungsi buatan sendiri (User-defined function)} \\
    Fungsi ini dibuat oleh programmer sendiri sesuai kebutuhan.
\end{enumerate}

\subsection{Membuat fungsi}

Untuk membuat fungsi kita bisa menuliskan:
{
\captionsetup[lstlisting]{labelformat=empty, justification=raggedright, singlelinecheck=false} %agar caption tanpa label dan di kiri
\begin{lstlisting}[language=c, caption={syntax}]
    tipe_data nama_fungsi(parameter){
    // kode yang akan dijalankan ketika fungsi dipanggil
    }
\end{lstlisting}
}
\begin{enumerate}[label={}, leftmargin=*]
    \item \verb|tipe_data|: tipe data untuk nilai kembalian fungsi (jika tidak ada nilai kembali bisa menggunakan tipe data void).
    \item \verb|nama_fungsi|: nama dari fungsi.
    \item \verb|parameter|: nilai atau variabel yang dimasukkan ke dalam fungsi.
\end{enumerate}

\subsubsection{Deklarasi dan definisi fungsi}

Sebuah fungsi terdiri dari dua bagian, yaitu:
Deklarasi: berisi tipe data nilai kembalian, nama fungsi, dan parameter.
Definisi: berisi kode yang akan dijalankan ketika fungsi dipanggil.
Contoh:
\begin{lstlisting}[language=c]
	void fungsi(int angka){ // Deklarasi
	// kode (Definisi)
	}
\end{lstlisting}
Pada contoh di atas, deklarasi dan definisi ditulis bersamaan.
Kita juga bisa menulis deklarasi dan definisi secara terpisah agar kode lebih mudah terbaca.
Untuk melakukannya kita tulis deklarasi di atas fungsi main(), dan definisi di bawah fungsi main().
Contoh:
\begin{lstlisting}[language=c]
	// Deklarasi
	void fungsi(int angka);

	//fungsi main
	int main(){

	}

	// Definisi
	void fungsi(int angka){

	}
\end{lstlisting}

\subsection{Memanggil fungsi}

Untuk menggunakan fungsi yang sudah dibuat, kita perlu memanggil fungsi tersebut.
Untuk memanggil sebuah fungsi, tulis nama fungsinya dan tanda kurung "()".
Contoh:
\begin{lstlisting}[language=c]
	void halo(){
		printf("Halo dunia!");
	}

	int main(){
		halo();
		return 0;
	}
\end{lstlisting}
\begin{verbatim}
	Output:
	Halo dunia!
\end{verbatim}
Sebuah fungsi bisa dipanggil berkali-kali. Contoh:
\begin{lstlisting}[language=c]
	void halo(){
		printf("Halo dunia!\n");
	}

	int main(){
		halo();
		halo();
		halo();
		return 0;
	}
\end{lstlisting}
\begin{verbatim}
	Output:
	Halo dunia!
	Halo dunia!
	Halo dunia!
\end{verbatim}

\subsection{Parameter dan Argumen}

Parameter adalah variabel yang didefinisikan ketika deklarasi atau definisi.
Parameter digunakan agar fungsi bisa menerima data dari luar fungsi.
Contoh:
\begin{lstlisting}[language=c]
	void jumlah(int a, int b){
		printf("%d", a+b);
	}
\end{lstlisting}
Fungsi di atas memiliki parameter dua angka integer a dan b.
Kedua parameter ini bisa digunakan di dalam fungsi, dalam contoh di atas dilakukan operasi penjumlahan dan menampilkan hasilnya.
\\ Nilai yang dikirimkan ke parameter adalah sebuah argumen.
Contoh:
\begin{lstlisting}[language=c]
	void jumlah(int a, int b){
		printf("%d", a+b);
	}

	int main(){
		jumlah(10, 5);

		return 0;
	}
\end{lstlisting}
\begin{verbatim}
	Output:
	15
\end{verbatim}
Dari contoh di atas, int a dan int b merupakan parameter, 10 dan 5 adalah argumen.
Ketika fungsi di atas dipanggil, nilai dari parameter a adalah 10, dan b adalah 5, sehingga menghasilkan output 15.

\subsection{Nilai kembalian}

Nilai kembalian adalah hasil yang dikirimkan oleh sebuah fungsi kepada bagian program yang memanggil fungsi tersebut.
Nilai kembalian harus sesuai dengan tipe data dari fungsi itu sendiri.
Fungsi dengan tipe void tidak memiliki kembalian.
Nilai kembalian ditentukan dengan keyword return.
contoh:
\begin{lstlisting}[language=c]
#include <stdio.h>

int tambah(int a, int b) {
	return a + b;  // mengembalikan hasil penjumlahan
}

int main(){
	int a = 5, b = 10;
	int jumlah = tambah(a, b); // nilai kembalian disimpan ke variabel
	printf("%d", jumlah);
}
\end{lstlisting}
Dari contoh diatas, nilai kembalian dari fungsi tambah() adalah penjumlahan kedua parameternya.
Nilai kembalian tersebut bisa didapatkan ketika fungsi dipanggil seperti pada baris 8.

\subsection{Scope variabel}

Dalam pemrograman, variabel memiliki scope atau jangkauan dimana variabel bisa diakses.
Contoh:
\begin{lstlisting}[language=c]
	void nilai(){
		int x = 1;
	}

	int main(){
		nilai();
		printf("%d", x);

		return 0;
	}
\end{lstlisting}
Dari contoh di atas, meskipun variable x sudah dideklarasi di dalam fungsi nilai().
Ketika variabel tersebut diakses di luar fungsi, akan terjadi error karena variabel x hanya bisa diakses di dalam fungsi.

\subsubsection{Variabel lokal}

Variabel lokal adalah variabel yang dideklarasikan di dalam fungsi atau sebuah blok kode (di dalam tanda kurung kurawal '{}').
Variabel ini hanya bisa diakses blok kode tempat variabel tersebut dideklarasikan.
Contoh:
\begin{lstlisting}[language=c]
#include <stdio.h>

int main(){
	int x = 10; //deklarasi variabel x

	if(x>9){
		int y = 10; //deklarasi variabel y`
		x = 0; // akses variabel x
	}
	printf("%d\n", x); // akses variabel x
	printf("%d\n", y); // akses variabel y.

	return 0;
}
\end{lstlisting}
Dari contoh di atas, variabel x dideklarasikan di dalam fungsi main, sehingga di dalam fungsi main().
Jadi variabel x bisa kita akses seperti pada baris 8 dan 10, karena masih di dalam fungsi main().
Sedangkan variabel y dideklarasikan di dalam fungsi if().
Jadi jika kita akses seperti pada baris 11 akan terjadi error, karena baris 11 berada di luar fungsi if().

\subsubsection{Variabel Global}

Variabel global adalah variabel yang dideklarasikan di luar semua fungsi.
Variabel ini bisa diakses dari semua tempat di dalam program.
Contoh:
\begin{lstlisting}[language=c]
#include <stdio.h>

int x = 1; // deklarasi global

void ubah(){
	x = 5; // akses di dalam fungsi, akan dieksekusi ketika fungsi dipanggil
}

int main(){
	printf("%d\n", x); // akses di main()
	x = 2;
	printf("%d\n", x);

	ubah(); // akses variabel x dari dalam fungsi.

	printf("%d\n", x);
	return 0;
}
\end{lstlisting}
\begin{verbatim}
	Output:
	1
	2
	5
\end{verbatim}
Dari contoh di atas, variabel x adalah variabel global yang mana bisa diakses dari fungsi main() maupun fungsi ubah().

\subsubsection{Penamaan variabel}

Jika variabel global dan variabel lokal memiliki nama yang sama, 
bahasa C akan menganggap kedua variabel tersebut sebagai dua variabel yang berbeda.
Contoh:
\begin{lstlisting}[language=c]
#include <stdio.h>

int x = 1; deklarasi global

void angka(){
	int x = 5; // deklarasi lokal
	printf("%d\n", x); // akses variabel lokal
}

int main(){
	angka();

	printf("%d\n", x); // akses variabel global

	return 0;
}
\end{lstlisting}
\begin{verbatim}
    Output:
    5
    1
\end{verbatim}
Variabel x dideklarasikan secara global dan secara lokal di dalam fungsi angka().
Sehingga ketika mengakses variabel x di dalam fungsi angka(), yang diakses adalah variabel lokal, bukan global.

\subsection*{Tugas Pendahuluan 1}
\begin{enumerate}
    \item Fungsi dapat membuat program lebih efisien dan mudah dibaca, jelaskan alasannya!
    \item Apa itu variabel statis? Jelaskan apa saja kegunaannya!
    \item Berikan contoh masalah yang akan lebih mudah diselesaikan dengan menggunakan fungsi!
    \item Buatlah sebuah fungsi yang mengembalikan nilai akar sebuah angka! (jangan gunakan fungsi library).
\end{enumerate}

\section{Rekursi}

Rekursi adalah teknik dimana sebuah fungsi memanggil dirinya sendiri.

\subsection{Struktur Rekursi}

Rekursi terbagi menjadi dua bagian
\begin{enumerate}
    \item \textbf{Kasus dasar (base case)} \\
    Sebuah kondisi yang membuat rekursi berhenti.
    Digunakan agar tidak terjadi rekursi terus-menerus.
    \item \textbf{Kasus rekursi (recursive case)} \\
    Bagian dimana fungsi memanggil dirinya sendiri.
\end{enumerate}

\subsection{Contoh penggunaan rekursi}

\begin{lstlisting}[language=c]
#include <stdio.h>

int faktorial(int n) {
	if (n == 0 || n == 1)  // base case
	return 1;
	else                   // recursive case
	return n * faktorial(n - 1);
}

int main() {
	printf("5! = %d\n", faktorial(5));
	return 0;
}
\end{lstlisting}
Contoh di atas adalah penggunaan rekursi untuk menghitung nilai faktorial.
Fungsi faktorial(n) akan mengembalikan nilai n x faktorial(n-1), lalu faktorial(n-1) akan mengembalikan nilai n-1 x faktorial(n-2), dan seterusnya.
Hal ini akan berulang terus menerus hingga kondisi base case terpenuhi yaitu ketika nilai n adalah 0 atau 1.
\begin{verbatim}
    faktorial(n) = n x faktorial(n-1)
                 = n x n-1 x faktorial(n-2)
                 = n x n-1 x n-2 x faktorial(n-3)
                 = n x n-1 x n-2 x ... x 2 x faktorial(1)
                 = n x n-1 x n-2 x ... x 2 x 1
\end{verbatim}

\subsection*{Tugas Pendahuluan 2}
\begin{enumerate}
    \item Berikan contoh masalah apa yang lebih mudah diselesaikan menggunakan rekursi!
    \item Apa saja kelebihan dan kekurangan dari rekursi? jelaskan!
    \item Perhatikan kode berikut:
    \begin{lstlisting}[language=c]
	int faktorial(int n) {
		if (n == 0 || n == 1)  // base case
		return 1;
		else                   // recursive case
		return n * faktorial(n - 1);
	}
\end{lstlisting}
   Jika ketika fungsi dipanggil kita masukkan argumen berupa angka negatif, apa yang terjadi?
   Apa solusi yang bisa kamu berikan?
    \item Pelajarilah algoritma searching! Tuliskan apa saja yang kamu pelajari!
    \item Algoritma searching apa yang paling efisien? kenapa?
\end{enumerate}